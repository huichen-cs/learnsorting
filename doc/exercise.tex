\documentclass[11pt,letterpaper]{article}
%\renewcommand{\familydefault}{\rmdefault}
%\renewcommand{\familydefault}{\sfdefault}
%\usepackage{tgbonum}

\usepackage[top=1in,bottom=1in,left=1in,right=1in]{geometry}
\usepackage{amsfonts,amsbsy,amsmath,amssymb}
\usepackage{color}
\usepackage[dvipsnames]{xcolor}
\usepackage{cite}
\usepackage{url}
\usepackage[autostyle]{csquotes} 
\usepackage{paralist} % paralist then enumitem; otherwise, error
\usepackage{enumitem} % conflict with paralist
\setlist{leftmargin=*}

\usepackage{listings}
\lstset{
    basicstyle=\scriptsize,
    escapeinside={*@}{@*},
    breaklines=true
}

\PassOptionsToPackage{hyphens}{url}\usepackage{hyperref}
\hypersetup{
    colorlinks,
    citecolor=black,
    filecolor=black,
    linkcolor=black,
    urlcolor=black
}


\usepackage{fancyhdr}
\fancyhf{}
\lfoot[]{{\vskip -0.8em \color{gray} \tiny{Revision: a13b778fe72e
}}}
\rfoot[{\vskip -0.8em \color{gray} \tiny{Revision: a13b778fe72e
}}]{}
\cfoot{\thepage}
\lhead[Sorting Basics]{Bubble Sort \& Quick Sort}
\rhead[Bubble Sort \& Quick Sort]{Sorting Basics}
\pagestyle{fancy}

\fancypagestyle{firstpage}{%
    \fancyhf{}% Clear header/footer
    \lfoot[]{{\vskip -0.8em \color{gray} \tiny{Revision: a13b778fe72e
}}}
    \rfoot[{\vskip -0.8em \color{gray} \tiny{Revision: a13b778fe72e
}}]{}
    \cfoot{\thepage}
    \renewcommand{\headrulewidth}{0pt}%
}

\makeatletter
\def\@maketitle{%
  \newpage
  \null
  \begin{center}%
  \let \footnote \thanks
      {\vspace{-4em} \LARGE \@title \par}%
  \vspace{-0.5em}
  \line(1, 0){200}  \\ \vspace{-1em}
  \line(1, 0){200}
    \vskip 1.5em%
    {\small
      \lineskip .5em%
      \begin{tabular}[t]{c}%
        \@author\\%
        \@date%
      \end{tabular}%
      }%
  \end{center}%
  \par
}
\makeatother

\title{Bubble Sort and Quick Sort} 
\author{Hui Chen (huichen@ieee.org)}
\date{February, 2017}
\begin{document}
\maketitle
\thispagestyle{firstpage}

\section{Objective}
Via experimenting with an implementation of the Bubble Sort and Quick Sort
algorithms in Python, at the conclusion of the discussion in class and this
exercise, you are expected to be able to explain and implement the Bubble Sort
and Quick Sort algorithms, to understand important sorting algorithm evaluation
criteria, in particular, to explain time complexity of the Bubble Sort and
Quick Sort algorithms, and to explain the concept of the divide-and-conquer
strategy using the design of the Quick Sort algorithm.

\section{Resources}

This document is available on the Web at the following URL,

\begin{itemize}
    \item \url{https://goo.gl/jPnkBf}
\end{itemize}

The associated presentation slides of these exercises are available at
the following URL,

\begin{itemize}
    \item \url{https://goo.gl/dQxUqQ}
\end{itemize}


You may find the source code of the Bubble Sort and Quick Sort implementation
written in Python at a \texttt{Github} repository via the following URL,

\begin{itemize}
    \item \url{https://goo.gl/phTjrS}
\end{itemize}

\noindent If you have \texttt{Eclipse} and have also installed the
\texttt{PyDev} plugin for \texttt{Eclipse}, you may directly import the project
into \texttt{Eclipse} after you have cloned the project using \texttt{git} or
the \texttt{Github Desktop} application. 

Alternatively, you can also find most of the source code at \texttt{Repl.it} at
the following URL,

\begin{itemize}
    \item \url{https://goo.gl/QA3pNO}
\end{itemize}

\noindent \texttt{Repl.it} is a simple Web-based IDE. To use the simple Web-based IDE, you
only need a Web browser, such as, the \texttt{FireFox} or \texttt{Chrome} Web
browser.  

Except Exercise~\ref{q:run} in the Exercises section below, you are
advised to create a free account at \texttt{Repl.it}, create a session of
yourself, copy the code to your own session at \texttt{Repl.it}, and work
on the excises in your own session, if you 
choose to use \texttt{Repl.it} instead of \texttt{Eclipse}.

\section{Exercises}

Complete the following exercises. When completing the exercises, think about
and try to explain {\em why} it happened besides observing {\em what} happened. 

\begin{enumerate}
    \item \label{q:run} Run the main program, observe the relationship between
        the running time and the size of the lists, and compare the observations
        between Bubble Sort and Quick Sort. 

    \item \label{q:sorted} In Exercise~\ref{q:run}, the integers in the lists
        that you are sorting are randomly generated. In this task, you are to
        generate ordered lists of numbers by revising the provided code. In
        particular, you will observe the relationship between running time and
        the size of the lists for the following two cases. 

        \begin{enumerate}
            \item Sort lists of integers that are ordered in an ascending
                order.  You may generate a list of integers in an ascending
                order in Python as follows,

                \begin{verbatim}
                my_list = list(range(0, 1000, 1))
                \end{verbatim}

            \item Sort lists of integers that are ordered in a descending
                order.  You may generate a list of integers in a descending
                order in Python as follows,

                \begin{verbatim}
                my_list = list(range(1000, 0, -1))
                \end{verbatim}

        \end{enumerate}
        For the two cases, compare the observations between Bubble Sort and
        Quick Sort, and compare the observations in this exercise with 
        those in Exercise~\ref{q:run}.


    \item Revise your code to select the middle member of the list as the
        pivot. Repeat the task specified in Exercise~\ref{q:sorted}. If the
        length of the list is {\em odd}, you can select the middle member of
        the list using its index as follows,

                \begin{verbatim}
                middle_member = my_list[(len(my_list)-1)/2]
                \end{verbatim}

        Note that in Python lists are indexed from $0$. Compare your observations
        in this exercise with those in Exercise~\ref{q:sorted}.

\end{enumerate}

\section{Acknowledgement}

The discussion in class and these exercises are inspired by the following
resources. 

\begin{itemize}

    \item The CS Unplugged project.
        \begin{itemize}
            \item \url{http://csunplugged.org/}
        \end{itemize}

    \item The OpenDSA project.
        \begin{itemize}
            \item \url{https://opendsa-server.cs.vt.edu/}
        \end{itemize}

    \item The Runestone Interactive project, in particular, the interactive
        online book of ``Problem Solving with Algorithms and Data Structures
        using Python'',

        \begin{itemize}
            \item \url{http://interactivepython.org/runestone/static/pythonds/index.html}
        \end{itemize}

\end{itemize}
You are encouraged to explore those resources. 


\end{document}
